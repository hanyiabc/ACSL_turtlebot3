\documentclass[12]{article}

\usepackage{hyperref}
\usepackage{graphicx}

\begin{document}

\includegraphics[width = \linewidth]{ACSL_logo.jpg}


This is a quick bring up guide for the turtle bot 3 waffle pi model, and the remote PC. It will closely follow the official Robotis E-Manual. \\\\
\url{https://emanual.robotis.com/docs/en/platform/turtlebot3/overview/} 
with notable  divergences covered in detail.







\newpage




1. To begin you will need to install the appropriate ROS version on your remote PC. The remote PC will be running ROS Melodic which is backwards compatible with ROS Kinetic (The Distribution we will flash on the Turtlebot's on board computer).

The following page from the ROBOTIS E manual describes the process for adding and installing all necessary packages to your ROS Environment.

\url {https://emanual.robotis.com/docs/en/platform/turtlebot3/pc_setup/#network-configuration} \\\\

2. The next step is to configure network settings for permanent use. The Remote PC will be controlling the turtlebot pc via wifi using SSH protocol. Steps for configuring the wifi on the remote pc can be found here.

\url {https://emanual.robotis.com/docs/en/platform/turtlebot3/pc_setup/#install-ros-1-on-remote-pc} \\

For this project we will use the Linksys Router on the \textbf{Linksys04294} network. It is helpful to write down the IP address of the remote pc, since you will need it later to setup the turtlebot pc network settings.\\\\

3. Now it is time to look at the turtlebot SBC. For the turtlebot 3 Waffle Pi model, the on board PC is a raspberry PI. This step requires a micro SD card with adapter, a monitor with an HDMI input, a USB keyboard and mouse, and a power source for the turtlebot. ROBOTIS provides a prebuilt desktop environment for the turtlebot with ROS kinetic. Instructions for flashing this distribution can be found here (\textbf{Step 6.2.1.2 Install Linux Based on Raspbian, Do not use the other two methods.}).\\

\url {https://emanual.robotis.com/docs/en/platform/turtlebot3/raspberry_pi_3_setup/#raspberry-pi-3-setup}




\end{document}
